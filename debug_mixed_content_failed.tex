\documentclass[11pt]{article}
\usepackage{xcolor}
\usepackage{multicol}
\usepackage{geometry}
\usepackage{xeCJK}
\usepackage{fancyvrb}
% 不使用 listings 包,避免 METAFONT 相关问题
% 
\usepackage{amsmath}
\usepackage{amsfonts}
\usepackage{amssymb}
\usepackage{graphicx}
\usepackage{url}
\usepackage{hyperref}
\hypersetup{colorlinks=false, pdfborder={0 0 0}, hidelinks}
% 不使用 booktabs 包,避免 METAFONT 相关问题
% 直接定义表格命令,避免依赖外部字体
\makeatletter
% 定义简单的表格规则命令,不依赖 booktabs
\newcommand{\toprule}{%
  \\[-0.5\baselineskip]\noindent\hrulefill\\[0.5\baselineskip]
}
\newcommand{\midrule}{%
  \\[0.5\baselineskip]\noindent\hrulefill\\[0.5\baselineskip]
}
\newcommand{\bottomrule}{%
  \\[0.5\baselineskip]\noindent\hrulefill\\[-0.5\baselineskip]
}
\newcommand{\cmidrule}[1]{%
  \\[0.5\baselineskip]\noindent\hrulefill\\[0.5\baselineskip]
}
\makeatother

% 页面尺寸 - 使用精确的边距设置
\geometry{
    paperwidth=400.0pt,
    paperheight=800.0pt,
    left=10pt,
    right=10pt,
    top=15pt,
    bottom=15pt,
    noheadfoot,
    includefoot=false,
    includehead=false
}

% 字体设置
% Using system default CJK fonts


% 行距设置
\renewcommand{\baselinestretch}{1.3}
\setlength{\baselineskip}{14.3pt}

% 定义 pandoc 需要的命令(如果未定义则定义)
\providecommand{\tightlist}{\setlength{\itemsep}{0pt}\setlength{\parskip}{0pt}}

% 三栏布局设置
\setlength{\columnsep}{8pt}
\setlength{\columnseprule}{0pt}

% 防止内容溢出和孤行/寡行
\widowpenalty=10000
\clubpenalty=10000
\raggedbottom
\tolerance=1000
\emergencystretch=3em

% 代码块样式(不使用 listings 包,改用简单的 verbatim 环境)
% \lstset 已移除,因为不再使用 listings 包
% 代码块将使用 verbatim 或 fancyvrb 环境处理

% 表格和浮动环境处理
\makeatletter
% 简化表格,避免在三栏中溢出
\renewenvironment{table}{}{}
\renewenvironment{table*}{}{}
% 定义 pandoc 需要的环境(代码块相关)
\newenvironment{Shaded}{}{}
\newenvironment{Highlighting}{}{}
% Longtable 支持 (Pandoc may generate longtable)
% xeCJK and multicol have compatibility issues with longtable
% We'll use the longtable package but it may not work perfectly in multicols
% Tables will be handled as best as possible
\usepackage{longtable}
\let\oldverbatim\verbatim
\let\oldendverbatim\endverbatim
\renewenvironment{verbatim}{\oldverbatim}{\oldendverbatim}
% 定义 Pandoc 代码高亮的 Token 命令(如果未定义则定义为空)
\providecommand{\BuiltInTok}[1]{#1}
\providecommand{\CommentTok}[1]{#1}
\providecommand{\ControlFlowTok}[1]{#1}
\providecommand{\DocumentationTok}[1]{#1}
\providecommand{\ErrorTok}[1]{#1}
\providecommand{\ExtensionTok}[1]{#1}
\providecommand{\FunctionTok}[1]{#1}
\providecommand{\ImportTok}[1]{#1}
\providecommand{\InformationTok}[1]{#1}
\providecommand{\KeywordTok}[1]{#1}
\providecommand{\NormalTok}[1]{#1}
\providecommand{\OperatorTok}[1]{#1}
\providecommand{\OtherTok}[1]{#1}
\providecommand{\PreprocessorTok}[1]{#1}
\providecommand{\RegionMarkerTok}[1]{#1}
\providecommand{\SpecialCharTok}[1]{#1}
\providecommand{\SpecialStringTok}[1]{#1}
\providecommand{\StringTok}[1]{#1}
\providecommand{\VariableTok}[1]{#1}
\providecommand{\VerbatimStringTok}[1]{#1}
\providecommand{\WarningTok}[1]{#1}
% 在多栏环境中,确保表格命令不使用 \noalign
% 这些命令已经在前面定义过了,这里不需要重复
% 处理链接
\let\href\url
\makeatother

\pagestyle{empty}
\begin{document}
\noindent
\begin{multicols*}{3}
\setlength{\parindent}{0pt}
\setlength{\parskip}{2pt}
\raggedright
% 在多栏环境中,\noalign 不能使用,需要重新定义为无操作
\makeatletter
\renewcommand{\noalign}[1]{
  % 在多栏环境中忽略 \noalign
}
\makeatother

\section{主标题}\label{ux4e3bux6807ux9898}

这是介绍段落。

\subsection{列表测试}\label{ux5217ux8868ux6d4bux8bd5}

\subsubsection{无序列表}\label{ux65e0ux5e8fux5217ux8868}

\begin{itemize}
\tightlist
\item
  项目1
\item
  项目2
\item
  项目3
\end{itemize}

\subsubsection{有序列表}\label{ux6709ux5e8fux5217ux8868}

\begin{enumerate}
\def\labelenumi{\arabic{enumi}.}
\tightlist
\item
  第一项
\item
  第二项
\item
  第三项
\end{enumerate}

\subsection{代码测试}\label{ux4ee3ux7801ux6d4bux8bd5}

行内代码: \texttt{print("hello")}

代码块:

\begin{Shaded}
\begin{Highlighting}[]
\KeywordTok{def}\NormalTok{ test():}
    \ControlFlowTok{return} \StringTok{"测试代码块内容不丢失"}
\end{Highlighting}
\end{Shaded}

\subsection{数学公式测试}\label{ux6570ux5b66ux516cux5f0fux6d4bux8bd5}

行内公式: \(E = mc^2\)

块公式: \[\int_0^1 x^2 dx = \frac{1}{3}\]

\subsection{表格测试}\label{ux8868ux683cux6d4bux8bd5}

{\def\LTcaptype{none} % do not increment counter
\begin{longtable}[]{@{}lll@{}}
\toprule
列1 & 列2 & 列3 \\
\midrule
\endhead
\bottomrule
\endlastfoot
A1 & B1 & C1 \\
A2 & B2 & C2 \\
\end{longtable}
}

\subsection{引用测试}\label{ux5f15ux7528ux6d4bux8bd5}

\begin{quote}
这是引用内容 多行引用测试
\end{quote}

\subsection{强调测试}\label{ux5f3aux8c03ux6d4bux8bd5}

\textbf{粗体文本} 和 \emph{斜体文本} 和 \textbf{\emph{粗斜体}}

\subsection{链接测试}\label{ux94feux63a5ux6d4bux8bd5}

\href{https://example.com}{链接文本}

\subsection{结尾标记}\label{ux7ed3ux5c3eux6807ux8bb0}

CONTENT\_END\_MARKER

\end{multicols*}
\end{document}
